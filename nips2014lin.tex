%%%%%%%%%%%%%%%%%%%%%%%%%%%%%%%%%%%%%%%%%
% Jacobs Landscape Poster
% LaTeX Template
% Version 1.1 (14/06/14)
%
% Created by:
% Computational Physics and Biophysics Group, Jacobs University
% https://teamwork.jacobs-university.de:8443/confluence/display/CoPandBiG/LaTeX+Poster
% 
% Further modified by:
% Nathaniel Johnston (nathaniel@njohnston.ca)
%
% This template has been downloaded from:
% http://www.LaTeXTemplates.com
%
% License:
% CC BY-NC-SA 3.0 (http://creativecommons.org/licenses/by-nc-sa/3.0/)
%
%%%%%%%%%%%%%%%%%%%%%%%%%%%%%%%%%%%%%%%%%


%------------------------------------------------------------------------------
%	PACKAGES AND OTHER DOCUMENT CONFIGURATIONS
%------------------------------------------------------------------------------

\documentclass[final]{beamer}

\usepackage[scale=1.24]{beamerposter} % Use the beamerposter package for laying out the poster
\usepackage{subfig}

\usetheme{confposter} % Use the confposter theme supplied with this template

\setbeamercolor{block title}{fg=ngreen!83,bg=white} % Colors of the block titles
\setbeamercolor{block body}{fg=black,bg=white} % Colors of the body of blocks
\setbeamercolor{block alerted title}{fg=white,bg=ngreen!70} % Colors of the highlighted block titles
\setbeamercolor{block alerted body}{fg=black,bg=ngreen!10} % Colors of the body of highlighted blocks
% Many more colors are available for use in beamerthemeconfposter.sty

% Use more stuff from tikz
\usetikzlibrary{arrows, backgrounds, positioning, fit}
\newcommand{\arclen}     {1.5cm}
\newcommand{\nodesize}   {1.8cm}
\newcommand{\nodepad}    {5pt}
\newcommand{\platepad}   {13pt}
\newcommand{\arcwidth}   {3pt}
\newcommand{\fieldwidth} {6pt}
\newcommand{\pointsize}  {4mm}

%-----------------------------------------------------------
% Define the column widths and overall poster size
% To set effective sepwid, onecolwid and twocolwid values, first choose how many columns you want and how much separation you want between columns
% In this template, the separation width chosen is 0.024 of the paper width and a 4-column layout
% onecolwid should therefore be (1-(# of columns+1)*sepwid)/# of columns e.g. (1-(4+1)*0.024)/4 = 0.22
% Set twocolwid to be (2*onecolwid)+sepwid = 0.464
% Set threecolwid to be (3*onecolwid)+2*sepwid = 0.708

\newlength{\sepwid}
\newlength{\onecolwid}
\newlength{\twocolwid}
\newlength{\threecolwid}
\setlength{\paperwidth}{48in} % A0 width: 46.8in
\setlength{\paperheight}{36in} % A0 height: 33.1in
\setlength{\sepwid}{0.024\paperwidth} % Separation width (white space) between columns
\setlength{\onecolwid}{0.22\paperwidth} % Width of one column
\setlength{\twocolwid}{0.464\paperwidth} % Width of two columns
\setlength{\threecolwid}{0.708\paperwidth} % Width of three columns
\setlength{\topmargin}{-0.5in} % Reduce the top margin size
%-----------------------------------------------------------

\usepackage{graphicx}  % Required for including images
\usepackage{booktabs} % Top and bottom rules for tables


%% Define bracket commands (normal, square and curly).
\newcommand{\brac} [1]  {\ensuremath{\left({#1}\right)}}
\newcommand{\sbrac}[1]  {\ensuremath{\left[{#1}\right]}}
\newcommand{\cbrac}[1]  {\ensuremath{\left\{{#1}\right\}}}
\newcommand{\abrac}[1]  {\ensuremath{\left\langle{#1}\right\rangle}}


%% Symbols

% General
\newcommand{\test}       {\ensuremath{^{*}}}
\newcommand{\testT}      {\ensuremath{^{*\top}\!}}
\newcommand{\ttest}      {\ensuremath{^{**}}}
\newcommand{\real}  [1] {\ensuremath{\mathbb{R}^{#1}}}
\newcommand{\ident} [1] {\ensuremath{\mathbf{I}_{#1}}}

% Variables
\newcommand{\lstate}    {\ensuremath{\mathbf{f}}}
\newcommand{\lcov}      {\ensuremath{\boldsymbol{\Sigma}}}
\newcommand{\obs}       {\ensuremath{\mathbf{y}}}
\newcommand{\hyper}     {\ensuremath{\boldsymbol{\theta}}}
\newcommand{\prmean}    {\ensuremath{\boldsymbol{\mu}}}
\newcommand{\prcov}     {\ensuremath{\mathbf{K}}}
\newcommand{\pomean}    {\ensuremath{\mathbf{m}}}
\newcommand{\pocov}     {\ensuremath{\mathbf{C}}}
\newcommand{\xcov}      {\ensuremath{\boldsymbol\Gamma_{\obs\pomean}}}
\newcommand{\Sobs}      {\ensuremath{\mathcal{Y}}}
\newcommand{\Sfunc}     {\ensuremath{\mathcal{M}}}
\newcommand{\scoef}     {\ensuremath{\kappa}}
\newcommand{\Sw}        {\ensuremath{w}}
\newcommand{\Kgain}     {\ensuremath{\mathbf{H}}}
\newcommand{\Linmat}    {\ensuremath{\mathbf{A}}}
\newcommand{\intcpt}    {\ensuremath{\mathbf{b}}}
\newcommand{\Fengy}     {\ensuremath{\mathcal{F}}}
\newcommand{\step}      {\ensuremath{\alpha}}
\newcommand{\jacob}[1]  {\ensuremath{\mathbf{J}_{#1}}}

% Augmented systems
\newcommand{\augobs}    {\ensuremath{\mathbf{z}}}
\newcommand{\augcov}    {\ensuremath{\mathbf{S}}}
\newcommand{\augLinmat} {\ensuremath{\mathbf{B}}}
\newcommand{\augintcpt} {\ensuremath{\mathbf{c}}}

% Gaussian Process
\newcommand{\obss}      {\ensuremath{y}}
\newcommand{\lstates}   {\ensuremath{f}}
\newcommand{\Lins}      {\ensuremath{a}}
\newcommand{\Linvec}    {\ensuremath{\mathbf{a}}}
\newcommand{\intcpts}   {\ensuremath{b}}
\newcommand{\inobs}     {\ensuremath{\mathbf{x}}}
\newcommand{\kernl}     {\ensuremath{k}}
\newcommand{\Kernl}     {\ensuremath{\mathbf{k}}}
\newcommand{\KERNL}     {\ensuremath{\mathbf{K}}}
\newcommand{\lvar}      {\ensuremath{\sigma^2}}
\newcommand{\lstd}      {\ensuremath{\sigma}}
\newcommand{\Lvar}      {\ensuremath{\boldsymbol\Lambda}}
\newcommand{\pomeans}   {\ensuremath{m}}
\newcommand{\pocovs}    {\ensuremath{C}}
\newcommand{\xcovs}     {\ensuremath{\Gamma}}
\newcommand{\khyper}    {\ensuremath{\theta}}
\newcommand{\khypers}   {\ensuremath{\boldsymbol\theta}}


%% Operations
\newcommand{\transpose}  {\ensuremath{^{\!\top}}}
\newcommand{\inv}        {\ensuremath{^{\text{-}1}}}
\newcommand{\deter}[1]   {\ensuremath{\left|{#1}\right|}}
\newcommand{\trace}[1]   {\ensuremath{\text{tr}\!\brac{#1}}}
\newcommand{\diag}[1]    {\ensuremath{\text{diag}\!\brac{#1}}}
\newcommand{\expec}[2]   {\ensuremath{\abrac{#2}_{\!{#1}}}}
\newcommand{\expece}[2]  {\ensuremath{\mathbb{E}_{#1}\!\sbrac{#2}}}
\newcommand{\evar} [2]   {\ensuremath{\mathbb{V}_{#1}\!\sbrac{#2}}}
\newcommand{\KL}[2]      {\ensuremath{\text{KL}\!\sbrac{{#1}\!\parallel\!{#2}}}}
\newcommand{\entropy}[1] {\ensuremath{\mathbb{H}\sbrac{#1}}}
\newcommand{\lnorm}[2]   {\ensuremath{\left\|{#2}\right\|_{{#1}}}}


%% Functions, PDFs etc
\newcommand{\nonlinf} {\ensuremath{g}}
\newcommand{\nonlin}[1] {\ensuremath{\nonlinf\!\brac{{#1}}}}
\newcommand{\nonlininv}[1] {\ensuremath{\nonlinf\inv\!\brac{{#1}}}}
\newcommand{\augnonlin}[1] {\ensuremath{h\!\brac{{#1}}}}
\newcommand{\prob}  [1] {\ensuremath{p\!\brac{#1}}}
\newcommand{\probC} [2] {\ensuremath{p\!\left({#1}\middle\vert{#2}\right)}}
\newcommand{\qrob}  [1] {\ensuremath{q\!\brac{#1}}}
\newcommand{\qrobC} [2] {\ensuremath{q\!\left({#1}\middle\vert{#2}\right)}}
\newcommand{\gaus}  [1] {\ensuremath{\mathcal{N}\!\brac{#1}}}
\newcommand{\gausC} [2] {\ensuremath{\mathcal{N}\!\left({#1}\middle\vert{#2}\right)}}
\newcommand{\bern}  [1] {\ensuremath{\textrm{Bern}\!\brac{#1}}}
\newcommand{\bernC} [2] {\ensuremath{\textrm{Bern}\!\left({#1}\middle\vert{#2}\right)}}
\newcommand{\kfunc} [2] {\ensuremath{\kernl\!\brac{{#1}, {#2}}}}
\newcommand{\expon} [2] {\ensuremath{{#1}\!\times\!10^{#2}}}


%% Operators
\DeclareMathOperator*{\argmax}{\operatorname*{argmax}}
\DeclareMathOperator*{\argmin}{\operatorname*{argmin}}


% Red underline
\newsavebox\MBox
\newcommand\Cline[2][red]{{\sbox\MBox{$#2$}%
  \rlap{\usebox\MBox}\color{#1}\rule[-1.6\dp\MBox]{\wd\MBox}{0.7pt}}}


%------------------------------------------------------------------------------
%	TITLE SECTION 
%------------------------------------------------------------------------------

\title{Extended and Unscented Gaussian Processes} % Poster title

\author{Daniel M.\ Steinberg\textsuperscript{1} and
        Edwin V.\ Bonilla\textsuperscript{2}} % Author(s)

\institute{\textsuperscript{1}NICTA, 
           \textsuperscript{2}The University of New South Wales} % Institution(s)

%------------------------------------------------------------------------------

\begin{document}

\addtobeamertemplate{block end}{}{\vspace*{2ex}} % White space under blocks
\addtobeamertemplate{block alerted end}{}{\vspace*{2ex}} % White space under highlighted (alert) blocks

\setlength{\belowcaptionskip}{2ex} % White space under figures
\setlength\belowdisplayshortskip{2ex} % White space under equations

\begin{frame}[t] % The whole poster is enclosed in one beamer frame

\begin{columns}[t] % The whole poster consists of three major columns, the second of which is split into two columns twice - the [t] option aligns each column's content to the top

\begin{column}{\sepwid}\end{column} % Empty spacer column

\begin{column}{\onecolwid} % The first column


%------------------------------------------------------------------------------
%	Problem statement and aims
%------------------------------------------------------------------------------

\begin{alertblock}{Inverse Problems}

Many problems in science and engineering have some \textbf{forward} or system
model, $\nonlin{\cdot}$, which is often nonlinear:
\begin{equation*}
    \obs = \nonlin{\lstate} + \boldsymbol\epsilon
\end{equation*}
\vspace{-2cm}
\begin{itemize}
    \item We can measure the outputs of the system, $\obs$, but the inputs,
        $\lstate$, are \textbf{latent}.
    \item We wish to infer these inputs \emph{without} access to the inverse
        system model, $\nonlininv{\cdot}$.
\end{itemize}

\end{alertblock}


\begin{block}{Aims}

\begin{enumerate}
    \item We require a posterior distribution over $\lstate$.
    \item We don't want to re-derive the
        learning equations for every new $\nonlin{\cdot}$.
    \item The gradients, $\partial{\nonlin{\lstate}} /
        \partial\lstate$, may not be known.
    \item Learning has to be fast (not MCMC).
    \item $\obs$ may be a continuous process, like the path swept out by a
        robotic arm, in which case $\lstate$ may be a \textbf{Gaussian
            Process}.
\end{enumerate}

\end{block}


%------------------------------------------------------------------------------
%	INTRODUCTION
%------------------------------------------------------------------------------

\begin{block}{Nonlinear Gaussian Models}

Prior over latent inputs $\lstate$,
\begin{equation}
    \prob{\lstate} = \gausC{\lstate}{\prmean, \prcov}
\end{equation}
Likelihood -- encodes $\obs = \nonlin{\lstate} + \boldsymbol{\epsilon}$,
\begin{equation}
    \probC{\obs}{\lstate} = \gausC{\obs}{\Cline{\nonlin{\lstate}}, \lcov}
\end{equation}%
Inversion = posterior,
\begin{equation}
    \probC{\lstate}{\obs} = \frac{\probC{\obs}{\lstate}\prob{\lstate}}
        {\prob{\obs}}
    \label{eq:post}
\end{equation}
But the marginal likelihood is intractable,
\begin{equation}
    \prob{\obs} = \int \gausC{\obs}{\Cline{\nonlin{\lstate}}, \lcov}
        \gausC{\lstate}{\prmean, \prcov} d\lstate.
\end{equation}

\end{block}


%------------------------------------------------------------------------------
%	Nonlinear Gaussian Models
%------------------------------------------------------------------------------

\vspace{-1.5cm}
\begin{figure}
    \subfloat[][]{
        \vspace{-2cm}
        \pgfdeclarelayer{Ilayer}
\pgfsetlayers{Ilayer,main}
%
\begin{tikzpicture}
    [	
	    font=\small, 
		node distance=1.5cm,
		odist/.style ={circle, draw=black!60, thick, inner sep=5pt, 
                        minimum size=1cm, fill=black!25},
		pdist/.style ={circle, draw=black!60, dashed, thick, inner sep=5pt, 
                        minimum size=1cm, fill=black!25},
		dist/.style  ={circle, draw=black!60, thick, inner sep=5pt, 
                        minimum size=1cm},
		point/.style ={circle,draw=black!60,fill=black!60,thick,inner sep=0pt,
                       minimum size=2mm},
		oarc/.style  ={->,>=latex',thick},
		plate/.style ={draw=black!60,rounded corners,thick,inner sep=13pt},
		reps/.style  ={anchor=south east,inner sep=3pt}
	]

    \node[dist]     (q0)                         {$\lstate$};

    \node[point]    (mean)  [above=of q0,xshift=-1cm,label=$\prmean$] {};

    \node[point]    (cov)   [above=of q0,xshift=1cm,label=$\prcov$]  {};

    \node[odist]    (x0)    [below=of q0]        {$\obs$};

    \node[point]    (lcov)  [right=of x0,label=$\lcov$]  {};

    \draw[oarc]     (q0) to (x0);
    \draw[oarc]     (mean) to (q0);
    \draw[oarc]     (cov) to (q0);
    \draw[oarc]     (lcov) to (x0);
    
\end{tikzpicture}

        \label{sfig:gaus}
    }
    \subfloat[][]{
        
\pgfdeclarelayer{Ilayer}
\pgfsetlayers{Ilayer,main}
%
\begin{tikzpicture}
    [	
	    font=\footnotesize, 
		node distance=5mm,
		odist/.style ={circle, draw=black!60, thick, inner sep=1pt, 
                        minimum size=6mm, fill=black!25},
		pdist/.style ={circle, draw=black!60, dashed, thick, inner sep=1pt, 
                        minimum size=6mm, fill=black!25},
		dist/.style  ={circle, draw=black!60, thick, inner sep=1pt, 
                        minimum size=6mm},
		point/.style ={circle,draw=black!60,fill=black!60,thick,inner sep=0pt,
                       minimum size=2mm},
		oarc/.style  ={->,>=latex',thick},
		ofeild/.style  ={-,>=latex',line width=0.8mm},
		plate/.style ={draw=black!60,rounded corners,thick,inner sep=13pt},
		reps/.style  ={anchor=south east,inner sep=3pt}
	]

    
    \node[dist]     (q0)                         {$\lstates_0$};
    \node[dist]     (q1)    [right=of q0]        {$\lstates_1$};
    \node[dist]     (q2)    [right=of q1]        {$\lstates_2$};
    \node[dist]     (q3)    [right=of q2]        {$\lstates_3$};
    \node           (d2)    [right=of q3]        {$\ldots$};

    \node[point]     (i0)    [above=of q0, label=$\inobs_0$]    {};
    \node[point]     (i1)    [above=of q1, label=$\inobs_1$]    {};
    \node[point]     (i2)    [above=of q2, label=$\inobs_2$]    {};
    \node[point]     (i3)    [above=of q3, label=$\inobs_3$]    {};
    \node            (d1)    [above=of d2, yshift=4mm]          {$\ldots$};

    \node[odist]    (x0)    [below=of q0]        {$\obss_{0}$};
    \node[odist]    (x1)    [right=of x0]        {$\obss_{1}$};
    \node[odist]    (x2)    [right=of x1]        {$\obss_{2}$};
    \node[odist]    (x3)    [right=of x2]        {$\obss_{3}$};
    \node           (d3)    [right=of x3]        {$\ldots$};
    
    \node[point]    (lcov)  [below=of x1,xshift=5mm,label=below:$\lstd$] {};

    \draw[ofeild]     (q0) to (q1);
    \draw[ofeild]     (q1) to (q2);
    \draw[ofeild]     (q2) to (q3);
    \draw[ofeild]     (q3) to (d2);

    \draw[oarc]     (i0) to (q0);
    \draw[oarc]     (i1) to (q1);
    \draw[oarc]     (i2) to (q2);
    \draw[oarc]     (i3) to (q3);

    \draw[oarc]     (q0) to (x0);
    \draw[oarc]     (q1) to (x1);
    \draw[oarc]     (q2) to (x2);
    \draw[oarc]     (q3) to (x3);

    \draw[oarc]     (lcov) to (x0);
    \draw[oarc]     (lcov) to (x1);
    \draw[oarc]     (lcov) to (x2);
    \draw[oarc]     (lcov) to (x3);
    
\end{tikzpicture}


        \label{sfig:gp}
    }
    \caption{Graphical model of nonlinear Gaussian \subref{sfig:gaus} and
        Gaussian process \subref{sfig:gp} inversion problems.}
\end{figure}


%----------------------------------------------------------------------------------------

\end{column} % End of the first column

\begin{column}{\sepwid}\end{column} % Empty spacer column

\begin{column}{\twocolwid} % Begin a column which is two columns wide (column 2)

\begin{columns}[t,totalwidth=\twocolwid] % Split up the two columns wide column

\begin{column}{\onecolwid}\vspace{-.6in} % The first column within column 2 (column 2.1)


%------------------------------------------------------------------------------
%	Inference and Linearization
%------------------------------------------------------------------------------

\begin{block}{Variational Inference \& Linearization}

We cannot find the posterior in \eqref{eq:post} because the of the intractable
marginal likelihood. In variational inference we can \emph{approximate} the
posterior, $\probC{\lstate}{\obs} \approx \qrob{\lstate} = \gausC{\lstate}
{\pomean, \pocov}$, and then establish as \emph{lower bound} on the log
marginal likelihood,
\begin{equation}
    \Fengy = \expec{q\lstate}{\log \probC{\obs}{\lstate}}
        - \KL{\qrob{\lstate}}{\prob{\lstate}}.
    \label{eq:fengy}
\end{equation}
Then can use $\Fengy$ to \emph{optimise} our posterior parameters:
\begin{equation}
    \pomean^* = \argmax_\pomean{\Fengy}, \qquad
    \pocov^* = \argmax_\pocov{\Fengy},
\end{equation}
where $\qrob{\lstate}$ should approach the true posterior
$\probC{\lstate}{\obs}$. \emph{But}, the expected log likelihood has an
intractable quadratic term,
\begin{equation*}
    \expec{q\lstate}{\log \probC{\obs}{\lstate}} = 
    -\frac{1}{2} \expec{q\lstate}{
        \brac{\obs - \Cline{\nonlin{\lstate}}}\transpose \lcov\inv
        \brac{\obs - \Cline{\nonlin{\lstate}}}} + \ldots
\end{equation*}
We use two tricks to get around this problem;
\begin{enumerate}
    \item we \emph{linearise} $\nonlin{\lstate}$,
        \begin{equation}
            \nonlin{\lstate} \approx \Linmat\lstate + \intcpt.
            \label{eq:linearize}
        \end{equation}
    \item we use a \emph{Newton method} to find the posterior mean with the
        linearized $\tilde\Fengy$,
        \begin{equation}
            \pomean_{k+1} = \pomean_k -
            \step\brac{\nabla_\pomean\nabla_\pomean\tilde\Fengy}\inv 
                \nabla_\pomean\tilde\Fengy,
            \label{eq:newton}
        \end{equation}
\end{enumerate}

Posterion parameters:
\begin{align}
    \pomean_{k+1} &= \brac{1-\step}\pomean_k + \step\prmean 
        + \step\Kgain_k\brac{\obs - \intcpt_k - \Linmat_k\prmean},
        \label{eq:pomean} \\
    \Kgain_k &= \prcov\Linmat_k\transpose\brac{\lcov +
        \Linmat_k\prcov\Linmat_k\transpose}\inv, 
        \label{eq:kgain} \\
    \pocov &= \brac{\ident{D} - \Kgain\Linmat}\!\prcov
    \label{eq:pocov}
\end{align}

\end{block}

%----------------------------------------------------------------------------------------

\end{column} % End of column 2.1

\begin{column}{\onecolwid}\vspace{-.6in} % The second column within column 2 (column 2.2)

%----------------------------------------------------------------------------------------
%	METHODS
%----------------------------------------------------------------------------------------

\begin{block}{Methods}

Lorem ipsum dolor \textbf{sit amet}, consectetur adipiscing elit. Sed laoreet accumsan mattis. Integer sapien tellus, auctor ac blandit eget, sollicitudin vitae lorem. Praesent dictum tempor pulvinar. Suspendisse potenti. Sed tincidunt varius ipsum, et porta nulla suscipit et. Etiam congue bibendum felis, ac dictum augue cursus a. \textbf{Donec} magna eros, iaculis sit amet placerat quis, laoreet id est. In ut orci purus, interdum ornare nibh. Pellentesque pulvinar, nibh ac malesuada accumsan, urna nunc convallis tortor, ac vehicula nulla tellus eget nulla. Nullam lectus tortor, \textit{consequat tempor hendrerit} quis, vestibulum in diam. Maecenas sed diam augue.

\end{block}

%----------------------------------------------------------------------------------------

\end{column} % End of column 2.2

\end{columns} % End of the split of column 2 - any content after this will now take up 2 columns width

%----------------------------------------------------------------------------------------
%	IMPORTANT RESULT
%----------------------------------------------------------------------------------------

\begin{alertblock}{Important Result}

Lorem ipsum dolor \textbf{sit amet}, consectetur adipiscing elit. Sed commodo molestie porta. Sed ultrices scelerisque sapien ac commodo. Donec ut volutpat elit.

\end{alertblock} 

%----------------------------------------------------------------------------------------

\begin{columns}[t,totalwidth=\twocolwid] % Split up the two columns wide column again

\begin{column}{\onecolwid} % The first column within column 2 (column 2.1)

%----------------------------------------------------------------------------------------
%	MATHEMATICAL SECTION
%----------------------------------------------------------------------------------------

\begin{block}{Mathematical Section}

Nam quis odio enim, in molestie libero. Vivamus cursus mi at nulla elementum sollicitudin. Nam quis odio enim, in molestie libero. Vivamus cursus mi at nulla elementum sollicitudin.
  
\begin{equation}
E = mc^{2}
\label{eqn:Einstein}
\end{equation}

Nam quis odio enim, in molestie libero. Vivamus cursus mi at nulla elementum sollicitudin. Nam quis odio enim, in molestie libero. Vivamus cursus mi at nulla elementum sollicitudin.

\begin{equation}
\cos^3 \theta =\frac{1}{4}\cos\theta+\frac{3}{4}\cos 3\theta
\label{eq:refname}
\end{equation}

Nam quis odio enim, in molestie libero. Vivamus cursus mi at nulla elementum sollicitudin. Nam quis odio enim, in molestie libero. Vivamus cursus mi at nulla elementum sollicitudin.

\begin{equation}
\kappa =\frac{\xi}{E_{\mathrm{max}}} %\mathbb{ZNR}
\end{equation}

\end{block}

%----------------------------------------------------------------------------------------

\end{column} % End of column 2.1

\begin{column}{\onecolwid} % The second column within column 2 (column 2.2)

%----------------------------------------------------------------------------------------
%	RESULTS
%----------------------------------------------------------------------------------------

\begin{block}{Results}

\begin{figure}
\includegraphics[width=0.8\linewidth]{fig/placeholder.jpg}
\caption{Figure caption}
\end{figure}

Nunc tempus venenatis facilisis. Curabitur suscipit consequat eros non porttitor. Sed a massa dolor, id ornare enim:

\begin{table}
\vspace{2ex}
\begin{tabular}{l l l}
\toprule
\textbf{Treatments} & \textbf{Response 1} & \textbf{Response 2}\\
\midrule
Treatment 1 & 0.0003262 & 0.562 \\
Treatment 2 & 0.0015681 & 0.910 \\
Treatment 3 & 0.0009271 & 0.296 \\
\bottomrule
\end{tabular}
\caption{Table caption}
\end{table}

\end{block}

%----------------------------------------------------------------------------------------

\end{column} % End of column 2.2

\end{columns} % End of the split of column 2

\end{column} % End of the second column

\begin{column}{\sepwid}\end{column} % Empty spacer column

\begin{column}{\onecolwid} % The third column

%----------------------------------------------------------------------------------------
%	CONCLUSION
%----------------------------------------------------------------------------------------

\begin{block}{Conclusion}

Nunc tempus venenatis facilisis. \textbf{Curabitur suscipit} consequat eros non porttitor. Sed a massa dolor, id ornare enim. Fusce quis massa dictum tortor \textbf{tincidunt mattis}. Donec quam est, lobortis quis pretium at, laoreet scelerisque lacus. Nam quis odio enim, in molestie libero. Vivamus cursus mi at \textit{nulla elementum sollicitudin}.

\end{block}

%----------------------------------------------------------------------------------------
%	ADDITIONAL INFORMATION
%----------------------------------------------------------------------------------------

\begin{block}{Additional Information}

Maecenas ultricies feugiat velit non mattis. Fusce tempus arcu id ligula varius dictum. 
\begin{itemize}
\item Curabitur pellentesque dignissim
\item Eu facilisis est tempus quis
\item Duis porta consequat lorem
\end{itemize}

\end{block}

%----------------------------------------------------------------------------------------
%	REFERENCES
%----------------------------------------------------------------------------------------

\begin{block}{References}

%\nocite{*} % Insert publications even if they are not cited in the poster
\small{\bibliographystyle{unsrt}
\bibliography{nips2014lin}\vspace{0.75in}}

\end{block}

%----------------------------------------------------------------------------------------
%	ACKNOWLEDGEMENTS
%----------------------------------------------------------------------------------------

%\setbeamercolor{block title}{fg=red,bg=white} % Change the block title color

\begin{block}{Acknowledgements}

\footnotesize{\rmfamily{This research was supported by the Science Industry Endowment
        Fund (RP 04-174) Big Data Knowledge Discovery project. NICTA is funded
        by the Australian Government through the Department of Communications
        and the Australian Research Council through the ICT Centre of
        Excellence Program.}} \\

\end{block}

%----------------------------------------------------------------------------------------
%	CONTACT INFORMATION
%----------------------------------------------------------------------------------------

\setbeamercolor{block alerted title}{fg=white,bg=npurple!62} % Change the alert block title colors
\setbeamercolor{block alerted body}{fg=black,bg=white} % Change the alert block body colors

\begin{alertblock}{Contact Information}

\begin{itemize}
\item Web: \href{http://www.university.edu/smithlab}{http://www.university.edu/smithlab}
\item Email: \href{mailto:john@smith.com}{john@smith.com}
\item Phone: +1 (000) 111 1111
\end{itemize}

\end{alertblock}

\begin{center}
\begin{tabular}{ccc}
\includegraphics[width=0.3\linewidth]{fig/logo-nicta}
\hspace{3cm}
\includegraphics[width=0.3\linewidth]{fig/logo-unsw}
\end{tabular}
\end{center}

%----------------------------------------------------------------------------------------

\end{column} % End of the third column

\end{columns} % End of all the columns in the poster

\end{frame} % End of the enclosing frame

\end{document}
